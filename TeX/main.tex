\documentclass[12pt]{article}

\usepackage{amsmath}
\usepackage{amssymb}
\usepackage{anyfontsize}
\usepackage[toc,page]{appendix}
\usepackage{array}
\usepackage{authblk}
\usepackage{booktabs}
\usepackage{caption}
\usepackage{cellspace}
\usepackage{color}
\usepackage{enumerate}
\usepackage{enumitem}
\usepackage{fancyhdr}
\usepackage{float}
\usepackage{geometry}
\usepackage{graphicx}
\usepackage{listings}
\usepackage{multicol}
\usepackage{newtxtext, newtxmath}
\usepackage{physics} 
\usepackage{sectsty}
\usepackage{subfigure}
\usepackage{tabularx}
\usepackage{wallpaper}

% \CenterWallPaper{.2}{nthu-logo.jpg}

\renewcommand\thesection{\Roman{section}.}
\renewcommand\thesubsection{\normalsize \roman{subsection}.}
\sectionfont{\large \centering}
\subsectionfont{\centering \normalsize \emph}

\geometry{
    a4paper,
    total = {170mm, 257mm},
    left = 20mm,
    top = 20mm,
    }

\definecolor{dkgreen}{rgb}{0,0.6,0}
\definecolor{gray}{rgb}{0.5,0.5,0.5}
\definecolor{mauve}{rgb}{0.58,0,0.82}
\lstset{frame = tb,
        language = Python,
        aboveskip = 3mm,
        belowskip = 3mm,
        showstringspaces = False,
        columns = flexible,
        basicstyle = {\small\ttfamily},
        numbers = left,
        numberstyle = \tiny\color{gray},
        keywordstyle = \color{blue},
        commentstyle = \color{dkgreen},
        stringstyle = \color{mauve},
        breaklines = true,
        breakatwhitespace = true,
        tabsize = 4}


%%%%%%%%%%%%%%%%%%%%%%%%%%%%%%%%%%%%%%%%%%%%%%% Again, Don't change anything Above %%%%%%%%%%%%%%%%%%%%%%%%%%%%%%%%%%%%%%%%%%%%%%%


\begin{document}

\pagestyle{fancy}
    \fancyhf{} % clear all header and footer fields
    \renewcommand{\headrulewidth}{0pt} %clear the header line 
    \fancyhead[R]{
        \ifnum\value{page}>1 \includegraphics[width = .23\textwidth]{../figure/nthu-logo.png} 
        \else \fi
        }
    \fancyfoot[C]{\thepage}

% title and info
\title{{\vspace{-2.0cm}
        \normalsize Collider Physics - Final Report}\\
        \textbf{Atmospheric Neutrino Oscillation in Super-K}}
\author{Yuan-Yen Peng}
\affil{\emph{Dept. of Physics, NTHU}\\
        \emph{Hsinchu, Taiwan}}
\date{\today}
\thispagestyle{fancy}
\maketitle

\setlength{\columnsep}{0.03\textwidth}
\begin{multicols}{2}
    
\section{Neutrino Oscillation}
    We comprehend that the government is obligated to upgrade the whole prosperity level of society. Some stages, thereby, ought to be fulfilled; simultaneously, requiring some counterpoise with different hierarchies, inevitably. Some utilitarians may assert that it must quell the low class of people so as to haul up the entire echelon of humankind. We, however, cast doubt on this argument, reckoning that the wipe-out method is too arbitrary and transgressed as a viewpoint of humanity. Ergo, we want to use the numerical model to subvert utilitarianism's implementation.
    \subsection{Neutrino Oscillation}
    We comprehend that the government is obligated to upgrade the whole prosperity level of society. Some stages, thereby, ought to be fulfilled; simultaneously, requiring some counterpoise with different hierarchies, inevitably. Some utilitarians may assert that it must quell the low class of people so as to haul up the entire echelon of humankind. We, however, cast doubt on this argument, reckoning that the wipe-out method is too arbitrary and transgressed as a viewpoint of humanity. Ergo, we want to use the numerical model to subvert utilitarianism's implementation.
    
\section{Super Kamiokonde}
    me stable stages, and we will observe the population invisible hand” to interrupt the system; in other words, wipe out the low-class people, etc. Afterward, observe the analogical quantities, too.


\section{Results}
    In reality, the source usage rate depends on distance, constructed a concentric structure, for example, Beijing (Figure), Chicago, etc. Here, our artificially built city will likewise develop around the center of sources; thus, we adopt the concentric zone model to discuss the hierarchy distribution.
    
    % \begin{figure}[H]
    %     \centering 
    %     \includegraphics[width = 0.45\textwidth]{./RK4/0.png}
    %     \includegraphics[width = 0.45\textwidth]{./RK4/2.png} 
    % \end{figure}
    % \begin{figure}[H]
    %     \centering 
    %     \includegraphics[width = 0.45\textwidth]{./RK4/4.png}  
    %     \includegraphics[width = 0.45\textwidth]{./RK4/6.png}
    %     \includegraphics[width = 0.45\textwidth]{./RK4/8.png}
    %     \includegraphics[width = 0.45\textwidth]{./RK4/10.png} 
    %     \caption{This the normal cloud with RK4, from $t=0$ to $t=10$ and $dt=0.01$.}
    %     \label{RK4_cloud}
    % \end{figure}

\section{Outlook}


\begin{thebibliography}{3}

	%Each item starts with a \bibitem{} command and the details thereafter.
	
    % \bibitem{1} Quick facts \#3: Poisson noise, https://reurl.cc/XjR2v3
    % \bibitem{2} Changing the contrast and brightness of an image! - Theory, https://reurl.cc/Lm5aWK
    % \bibitem{3} Gaussian Approach to Poisson Noise - Adding Poisson noise to an image, \emph{slack overflow}, answer from \emph{Vimieiro}, edited by \emph{BOT}.
    % \bibitem{4} Shot noise, Wikipedia, https://reurl.cc/ZAN4Xg
    % \bibitem{5} Shot Noise, \emph{sunny lee}, https://reurl.cc/NAb9Vn 
        
    %%% The 1,2 etc. are used to cite in text. See up in the intro for an example
    %%% When you want to cite in your cite, type in \cite{} wherever you want
    
\end{thebibliography}

    
\end{multicols}


\end{document}