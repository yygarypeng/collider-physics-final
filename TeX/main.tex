\documentclass[12pt]{article}

\usepackage{amsmath}
\usepackage{amssymb}
\usepackage{anyfontsize}
\usepackage[toc,page]{appendix}
\usepackage{array}
\usepackage{authblk}
\usepackage{booktabs}
\usepackage{caption}
\usepackage{cellspace}
\usepackage{color}
\usepackage{enumerate}
\usepackage{enumitem}
\usepackage{fancyhdr}
\usepackage{float}
\usepackage{fontspec}
\usepackage{geometry}
\usepackage{graphicx}
\usepackage{listings}
\usepackage{multicol}
\usepackage{physics} 
\usepackage{sectsty}
\usepackage{subfigure}
\usepackage{unicode-math}
\usepackage{url}
\usepackage{tabularx}
\usepackage{wallpaper}
\usepackage{xeCJK}

\setmainfont{Times New Roman}
\setCJKmainfont{教育部標準宋體UN}
\setmonofont{Courier New}
\setmathfont{XITS Math}

\renewcommand\thesection{\Roman{section}.}
\renewcommand\thesubsection{\normalsize\roman{subsection}.}
\sectionfont{\centering\large}
\subsectionfont{\centering\normalsize\it}

\geometry{
    a4paper,
    total = {170mm, 257mm},
    left = 20mm,
    top = 20mm,
    }

\definecolor{dkgreen}{rgb}{0,0.6,0}
\definecolor{gray}{rgb}{0.5,0.5,0.5}
\definecolor{mauve}{rgb}{0.58,0,0.82}
\lstset{frame = tb,
        language = Python,
        aboveskip = 3mm,
        belowskip = 3mm,
        showstringspaces = False,
        columns = flexible,
        basicstyle = {\small\ttfamily},
        numbers = left,
        numberstyle = \tiny\color{gray},
        keywordstyle = \color{blue},
        commentstyle = \color{dkgreen},
        stringstyle = \color{mauve},
        breaklines = true,
        breakatwhitespace = true,
        tabsize = 4}


%%%%%%%%%%%%%%%%%%%%%%%%%%%%%%%%%%%%%%%%%%%%%%% Again, Don't change anything Above %%%%%%%%%%%%%%%%%%%%%%%%%%%%%%%%%%%%%%%%%%%%%%%


\begin{document}

\pagestyle{fancy}
    \fancyhf{} % clear all header and footer fields
    \renewcommand{\headrulewidth}{0pt} %clear the header line 
    \fancyhead[R]{
        \ifnum\value{page}>1 \includegraphics[width = .23\textwidth]{../figure/nthu-logo.png} 
        \else \fi
        }
    \fancyfoot[C]{\thepage}

% title and info
\title{{\vspace{-2.5cm}
        \normalsize Collider Physics - Final Report}\\
        \textbf{Atmospheric Neutrino Oscillation in Super-K}}
\author{Yuan-Yen Peng(彭元彥)}
\affil{\emph{Dept. of Physics, NTHU}\\
        \emph{Hsinchu, Taiwan}}
\date{\today}
\thispagestyle{fancy}
\maketitle

\setlength{\columnsep}{0.03\textwidth}
\begin{multicols}{2}
    
\section{Neutrino Oscillation}
    Four types of neutrino sources are classified as solar neutrino, atmospheric neutrino, accelerator neutrino, and reactor neutrino. Besides, these neutrino sources are investigated in different experimental facilities. As the aforementioned, the first affiliated neutrino observatory is SNO (Sudbury Neutrino Observatory) in Canada; the second representative neutrino observatory is Super-K (Super Kamiokande) in Japan; the third corresponding observatories are KEK-SuperK (Japan), T2K(Japan), etc; the last observatories representing are KamLAND, Daya bay (China), etc. In this final report, we are going to present and elaborate on atmospheric neutrino oscillation in Super-K.

    \subsection{Neutrino Oscillation}
        Neutrino oscillation is a phenomenon that describes the flavor of neutrino oscillating among three different flavors. In general, the ``flavor eigenstates'' are related to the ``mass eigenstates'' of the'' with $3 \times 3$ unitary mixing matrix:
        \[
            | \nu_{\alpha} \rangle = \sum_{i} U_{\alpha i} | \nu_{i} \rangle
        \]
        The matrix $U$ is the PMNS matrix (Pontecorvo-Maki-Nakagawa-Sakata matrix: mixing flavor eigenstates of leptons); $\nu_\alpha$ is the mass eigenstate and $\nu_i$ is the flavor eigenstate. Thus, we need the PMNS matrix in order to find the relation between eigenspaces. The PMNS matrix is \cite{SKexp} \cite{wiki}:
        \begin{align*}
            U_{\alpha i} &= 
            \begin{bmatrix}
                U_{e1}     & U_{e2}     & U_{e3}    \\
                U_{\mu 1}  & U_{\mu 2}  & U_{\mu 3} \\
                U_{\tau 1} & U_{\tau 2} & U_{\tau 3}\\
            \end{bmatrix}\\
            &=
            \begin{bmatrix}
                1 & 0       & 0     \\
                0 & c_{23}  & s_{23}\\
                0 & -s_{23} & c_{23}\\
            \end{bmatrix}
            \begin{bmatrix}
                c_{13}             & 0 & s_{13}e^{i\delta}\\
                0                  & 1 & 0                \\
                -s_{13}e^{i\delta} & 0 & c_{13}           \\
            \end{bmatrix}\\
            &\quad
            \begin{bmatrix}
                c_{12}  & s_{12} & 0\\
                -s_{12} & c_{12} & 0\\
                0       & 0      & 1\\
            \end{bmatrix}
        \end{align*}
        where $\alpha = e,\ \mu,\ \tau$; $i = 1,\ 2,\ 3$; $c_{ab}$ and $s_{ab}$ means $sin$ and $cos$ with mixing angle $\theta_{ab}$; $\delta$ means CP-violating phase (s.t. second term of the matrix is small). The first term of the matrix is corresponding to ``solar mixing'', the second term of the matrix is known to be small, and the third term of the matrix describes ``atmospheric mixing''.

        The experiment of testing the neutrino oscillation needs a distance $L$ to investigate. We, thence, need to consider the state propagating. For mass eigenstate 
        \[
            |\nu_{j}(t)\rangle = e^{-i(E_{j}t - \vec{p_{j}} \cdot \vec{x})} |\nu_{j}(0)\rangle
        \]
        take ultrarelativistic limit $|\vec{p_j}| \gg m_j$ and also drop out the relative phase. When $t\approx L$: \cite{wiki}
        \[
            |\nu_{j}(L)\rangle = e^{-i(\frac{m_{j}^{2}L}{2E})} |\nu_{j}(0)\rangle
        \]

        In the wake of knowing the propagation of the mass eigenstate, we want to know the ``interference'' between the flavor eigenstate. Therefore, suppose the initial lepton with the flavor eigenstate $\alpha$ and what is the probability if we observe $\beta$ when $t \approx L$ \cite{SKexp}
        \begin{align*}
            P_{\alpha \rightarrow \beta}&= |\langle\nu_{\beta}|\nu_{\alpha}\rangle|^{2}\\
                                        &= \Big| \sum_{j}U_{\alpha j}^{*}U_{\beta j}e^{-i(\frac{m_{j}^{2}L}{2E})} \Big|^2\\
                                        & \quad drop\ out\ \mathbb{Im}\\
                                        &\Rightarrow \delta_{\alpha \beta} - 4\sum_{j > i} \mathbb{Re} \Bigl\{ U_{\alpha i}^{*}U_  {\beta j}U_{\alpha k}U_{\beta k}^{*} \Bigr\}\\ 
                                        &\quad \times \sin^{2} \Big( \frac{\Delta_{jk}m^2L}{4E} \Big)
                                        \tag{1}\label{prob_raw}
        \end{align*}
    
        \subsection{One mass scale dominant approximation}
        The one mass scale dominant approximation means there is one mass among the three masses which is manifest larger than the others in the mass square difference. (This approximation has been supported by the neutrino experiments \cite{SKexp}) Here, we suppose $m_3$ is the dominant mass :
        \[
            |m_{2}^{2} - m_{1}^{2}| \ll |m_{3}^{2} - m_{1,\ 2}^{2}|
        \]
        Additionally, we can rewrite the phase inside the square of $\sin$ in equation (\ref{prob_raw}) as:
        \begin{align*}
            \frac{\Delta_{jk}(mc)_{2}L}{4 \hbar c} &= \frac{GeV\ fm}{4 \hbar c} \times \frac{\Delta_{jk} m^2}{eV^2} \frac{L}{km} \frac{GeV}{E}\\
            &\approx 1.27 \times \frac{\Delta_{jk} m^2}{eV^2} \frac{L}{km} \frac{GeV}{E}
        \end{align*}
        According to this method, the propagating probabilities in vacuum for atmospheric neutrinos are: \cite{SKexp}
        \begin{align*}
            P(\nu_{e} \rightarrow \nu_{\mu})    &= P(\nu_{\mu}\rightarrow \nu_{e})&\\
                                                &= \sin^{2}\theta_{23} \sin^{2}\theta_{13} \Big( \frac{1.27 \Delta m^{2} L}{E} \Big)&\\
            P(\nu_{e} \rightarrow \nu_{e})      &= 1 - \sin^{2}\theta_{13}\sin^{2} \Big( \frac{1.27\Delta m^{2} L}{E} \Big)\\ 
            P(\nu_{\mu} \rightarrow \nu_{\mu})  &= 1 - 4 \cos^{2}\theta_{13}\sin^{2}\theta_{23}\\
                                                &\times \big( 1 - \cos^{2}\theta_{13}\sin^{2} \theta_{23} \big) \Big( \frac{1.27\Delta m^{2} L}{E} \Big)\\
        \end{align*}


\section{Super Kamiokonde}


\section{Results}
    In reality, the source usage rate depends on distance, constructed a concentric structure, for example, Beijing (Figure), Chicago, etc. Here, our artificially built city will likewise develop around the center of sources; thus, we adopt the concentric zone model to discuss the hierarchy distribution.
    
    % \begin{figure}[H]
    %     \centering 
    %     \includegraphics[width = 0.45\textwidth]{./RK4/0.png}
    %     \includegraphics[width = 0.45\textwidth]{./RK4/2.png} 
    % \end{figure}
    % \begin{figure}[H]
    %     \centering 
    %     \includegraphics[width = 0.45\textwidth]{./RK4/4.png}  
    %     \includegraphics[width = 0.45\textwidth]{./RK4/6.png}
    %     \includegraphics[width = 0.45\textwidth]{./RK4/8.png}
    %     \includegraphics[width = 0.45\textwidth]{./RK4/10.png} 
    %     \caption{This the normal cloud with RK4, from $t=0$ to $t=10$ and $dt=0.01$.}
    %     \label{RK4_cloud}
    % \end{figure}

\section{Outlook}


\begin{thebibliography}{1}

	%Each item starts with a \bibitem{} command and the details thereafter.
	
    \bibitem{SKexp} J. Hosaka et al. Physical Review D (2006)
    \bibitem{wiki} Wikipedia Contributors, Neutrino oscillation, Wikipedia (2022).
    % \bibitem{3} Gaussian Approach to Poisson Noise - Adding Poisson noise to an image, \emph{slack overflow}, answer from \emph{Vimieiro}, edited by \emph{BOT}.
    % \bibitem{4} Shot noise, Wikipedia, https://reurl.cc/ZAN4Xg
    % \bibitem{5} Shot Noise, \emph{sunny lee}, https://reurl.cc/NAb9Vn 
        
    %%% The 1,2 etc. are used to cite in text. See up in the intro for an example
    %%% When you want to cite in your cite, type in \cite{} wherever you want
    
\end{thebibliography}

    
\end{multicols}


\end{document}